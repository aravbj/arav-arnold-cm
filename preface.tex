\chapter*{Preface}
These solutions rose out of a personal goal I had set for myself to finish the most mathematically concrete treatment of Classical Mechanics to date, Mathematical Methods of Classical Mechanics by V.A.Arnold. Being a graduate student working on galactic dynamics, I wanted to get a solid foundation on the bread and butter of my field. I was trained in physics without worrying too much about mathematical rigor and this exercise is my attempt at getting acquainted with the tools of differential geometry and its application to Hamiltonian systems. Further, a firm grasp of differential geometry is highly useful when learning general relativity and my goal after finishing Arnold's book is to finally tackle Wald's book on general relativity.\par
Working on galactic dynamics, almost all of the topics in this book are relevant to me except for the section on rigid body dynamics which I have skipped on my first reading. I may or may not get back to this chapter at a later time. Further, given that this is a completely solo effort made worse by my lack of experience with rigorous mathematical proofs, I am in no way claiming that my proofs are correct or as rigorous as they could be and I welcome corrections and suggestions from anyone taking their time to read this document. Working out these proofs as explicitly as possible helped me at least convince myself that a result is right, and as a physicist, this was my primary aim. This work is targeted mainly for physics students as a reference in case they are not accustomed with mathematical proofs and/or don't wish to spend their time trying to work out proofs, but would like to see what they look like.\par
I have tried my best to work things out as explicitly as possible using only the notations and results used in the book. In some sections one has to use results which are presented in further chapters. Since the book has a notoriously bad labeling system for problems, I have written out the problem statement along with the page number where the problem can be found in the book in parenthesis. Questions with self-explanatory answers that are provided in the text are not given different solutions. Notes or ideas that I found useful from different sources during this study are given as footnotes or in boxes where and when relevant. There is also a dedicated subsection proving the theorems of Cartan calculus. 